\documentclass[12pt, letterpaper]{article}

    \usepackage[utf8]{inputenc}
    \usepackage{array}
    \usepackage{amsmath, amssymb, amsthm}
    \usepackage{enumerate}
    \usepackage{newunicodechar}
    \usepackage{mathtools}
    \usepackage{bm}
    
    \DeclareUnicodeCharacter{2212}{-}
    
    \theoremstyle{remark}
    \newtheorem{problem}{Problem}
    
    \theoremstyle{remark}
    \newtheorem*{solution}{\textbf{Solution}}

    \newenvironment{amatrix}[1]{%
    \left[\begin{array}{@{}*{#1}{c}|c@{}}
    }{%
        \end{array}\right]
    }
    
    \title{Linear Algebra \\
        \large Practice Problems for Quiz III}
    \author{Andres Valdes}
    \date{April 8, 2018}
    
\begin{document}

\begin{titlepage}

	\maketitle
	\thispagestyle{empty}

\end{titlepage}

\section*{Introduction}

Herein we will discuss the solutions for problems
relating to the third assignment in linear algebra. The
goal of this is to become proficient in the skills required
to succeed in both the class and the practical applications
of it.

\bigskip
\noindent
\textit{\small *The book from which the problems are taken is
	Steve Leon's Linear Algebra with Applications, 9e.}

\begin{center}

	\section*{Problem Set}

	\begin{tabular}{|c|l|}

		\hline
		§   & \textbf{EXERCISES}  \\
		\hline\hline
		3.6 & 1, 4, 5, 12, 22, 25 \\
		\hline
		4.2 & 1, 2, 5, 6, 14, 17  \\
		\hline
		4.3 & 1, 2, 5, 7, 9, 11   \\
		\hline
		5.2 & 1, 2, 3, 4, 13      \\
		\hline
	\end{tabular}

\end{center}

\pagebreak

\section*{3.6\quad{}Row Space and Column Space}

\begin{problem}

For each of the following matrices, find a basis
for the row space, a basis for the column space,
and a basis for the null space.
\begin{enumerate}[(a)]

	\item \(
	      \begin{bmatrix*}[r]
		      1 & 3 & 2 \\
		      2 & 1 & 4 \\
		      4 & 7 & 8
	      \end{bmatrix*}
	      \)

	\item \(
	      \begin{bmatrix*}[r]
		      −3 & 1 & 3 & 4 \\
		      1 & 2 & −1 & −2 \\
		      −3 & 8 & 4 & 2
	      \end{bmatrix*}
	      \)

	\item \(
	      \begin{bmatrix*}[r]
		      1 & 3 & −2 & 1 \\
		      2 & 1 & 3 & 2 \\
		      3 & 4 & 5 & 6
	      \end{bmatrix*}
	      \)

\end{enumerate}

\end{problem}

\begin{solution}

	The solutions will be broken down into the basis, $B$, of
	the \textbf{row space} ($RS(A)$), \textbf{column space} ($CS(A)$), and
	\textbf{null space} ($N(A)$), in that
	order.

	\begin{enumerate}[(a)]
		\item Let \( A =
		      \begin{bmatrix*}[r]
			      1 & 3 & 2 \\
			      2 & 1 & 4 \\
			      4 & 7 & 8
		      \end{bmatrix*}
		      \)

		      \bigskip

		      I) \(B(RS(A))\), the basis of the row space of $A$, can be found by
		      first reducing the matrix to its row echelon form $U$ to
		      eliminate linearly dependent rows.

		      The row echelon form of \(A,\ U =
		      \begin{bmatrix*}[r]
			      1 & 3 & 2 \\
			      0 & 1 & 0 \\
			      0 & 0 & 0
		      \end{bmatrix*}
		      \).

		      This leaves two linearly independent rows, \((1, 3, 2),\ (0, 1, 0)\).
		      These two rows form the basis of the row space of $A$, as all other
		      rows of $A$ can be obtained through addition or scalar multiplication
		      of these rows.

		      \(\therefore B(RS(A)) = \left\{(1, 3, 2),(0, 1, 0)\right\}\).

		      \pagebreak

		      II) \(B(CS(A))\), the basis of the column space of $A$, can be found
		      by reducing $A$ to a row echelon form $U$ and obtaining from $U$ the
		      columns with leading ones. These columns of $A$ form \(B(CS(A))\).

		      The row echelon form of \(A,\ U =
		      \begin{bmatrix*}[r]
			      1 & 3 & 2 \\
			      0 & 1 & 0 \\
			      0 & 0 & 0
		      \end{bmatrix*}
		      \).

		      The columns with leading ones are \(\bm{c_1},\ \bm{c_2}\), meaning
		      that these are the columns from \(A\) that form \(B(CS(A))\).

		      \(\therefore B(CS(A)) = \left \{\begin{bmatrix*}[r]
			      1 \\
			      2 \\
			      4
		      \end{bmatrix*}, \begin{bmatrix*}[r]
			      3 \\
			      1 \\
			      7
		      \end{bmatrix*} \right \}.\)

		      \bigskip

		      III) \(B(N(A))\), the basis of the null space of $A$, is
		      the set of linearly independent vectors that span the entirety of
		      \(N(A)\), the set of vectors \(\bm{x}\) such that \(A\bm{x}=\bm{0}\).

		      First, we solve for \(N(A)\):
		      \[
			      \begin{amatrix}{3}
				      1 & 3 & 2 & 0 \\
				      2 & 1 & 4 & 0 \\
				      4 & 7 & 8 & 0
			      \end{amatrix} \sim
			      \begin{amatrix}{3}
				      1 & 0 & 2 & 0 \\
				      0 & 1 & 0 & 0 \\
				      0 & 0 & 0 & 0
			      \end{amatrix}
		      \]
		      And so it follows that \begin{align*}
			      x_1 + \phantom{x_2} + 2x_3 & = 0, \\
			      \phantom{x_1} + x_2        & = 0
		      \end{align*}
		      Let \(x_3 = \alpha\), where \(\alpha \in \mathbb{R}\),
		      then \(N(A) = \begin{bmatrix*}[r]
			      -2\alpha \\
			      0 \\
			      \alpha
		      \end{bmatrix*} = \alpha\begin{bmatrix*}[r]
			      -2 \\
			      0 \\
			      1
		      \end{bmatrix*}\)

		      \bigskip

		      \(\therefore B(N(A)) = \left\{\begin{bmatrix*}[r]
			      -2 \\
			      0 \\
			      1
		      \end{bmatrix*}\right\}\) as linear combinations of these
		      linearly independent vectors span the entirety of \(N(A)\).

		      \pagebreak

		\item Let \( A =
		      \begin{bmatrix*}[r]
			      −3 & 1 & 3 & 4 \\
			      1 & 2 & −1 & −2 \\
			      −3 & 8 & 4 & 2
		      \end{bmatrix*}
		      \)

		      I) Through row operations, we obtain the reduced row echelon
		      form of \(A\),

		      \[U^* = \begin{bmatrix*}[r]
				      1 & 0 & 0 & -\frac{10}{7} \\
				      0 & 1 & 0 & -\frac{2}{7} \\
				      0 & 0 & 1 & 0
			      \end{bmatrix*}\]

		      Where we find that all three rows are linearly independent.

		      \(\therefore B(RS(A)) = \left\{(1, 0, 0, -\frac{10}{7}), (0, 1, 0, -\frac{2}{7}), (0, 0, 1, 0)\right\}\)

		      \bigskip

		      II) Through the same reduced row echelon form, we can see
		      that the columns containing leading ones are \(\bm{c_1},\ \bm{c_2},\ \bm{c_3}\).

		      \(\therefore B(CS(A)) = \left\{\begin{bmatrix*}[r]
			      -3 \\
			      1 \\
			      -3
		      \end{bmatrix*},
		      \begin{bmatrix*}[r]
			      1 \\
			      2 \\
			      8
		      \end{bmatrix*},
		      \begin{bmatrix*}[r]
			      3 \\
			      -1 \\
			      4
		      \end{bmatrix*}\right\}\)

		      \bigskip

		      III) Solve for the \(N(A)\):

		      \[\begin{amatrix}{4}
				      1 & 0 & 0 & -\frac{10}{7} & 0 \\
				      0 & 1 & 0 & -\frac{2}{7} & 0 \\
				      0 & 0 & 1 & 0 & 0
			      \end{amatrix}\]

		      Let \(x_4 = \alpha,\ \alpha \in \mathbb{R}\), then \begin{align*}
			      x_1 + \phantom{x_2} \phantom{+} \phantom{x_3} \phantom{+} -\frac{10}{7}\alpha     & = 0, \\
			      \phantom{x_1} \phantom{+} x_2 + \phantom{x_3} \phantom{+} -\frac{2}{7}\alpha      & = 0, \\
			      \phantom{x_1} \phantom{+} \phantom{x_2} \phantom{+} x_3 \phantom{+} \phantom{x_4} & = 0
		      \end{align*}

		      And \(N(A) = \begin{bmatrix*}[r]
			      \frac{10}{7}\alpha \\
			      \frac{2}{7}\alpha \\
			      0 \\
			      \alpha
              \end{bmatrix*}. \)
              
              \(\therefore B(N(A)) = \left\{\begin{bmatrix*}[r]
			      \frac{10}{7} \\
			      \frac{2}{7} \\
			      0 \\
			      1
		      \end{bmatrix*}\right\}.\)


	\end{enumerate}

\end{solution}

\begin{problem}

In each of the following, determine the dimension
of the subspace of \(\mathbb{R}^3\) spanned by the given vectors.

\begin{enumerate}[(a)]
	\item \(
	      \begin{bmatrix*}[r]
		      1 \\
		      -2 \\
		      2
	      \end{bmatrix*},
	      \begin{bmatrix*}[r]
		      2 \\
		      -2 \\
		      4
	      \end{bmatrix*},
	      \begin{bmatrix*}[r]
		      -3 \\
		      3 \\
		      6
	      \end{bmatrix*}
	      \)
	\item \(
	      \begin{bmatrix*}[r]
		      1 \\
		      1 \\
		      1
	      \end{bmatrix*},
	      \begin{bmatrix*}[r]
		      1 \\
		      2 \\
		      3
	      \end{bmatrix*},
	      \begin{bmatrix*}[r]
		      2 \\
		      3 \\
		      1
	      \end{bmatrix*}
	      \)

\end{enumerate}

\end{problem}

\section*{4.2\quad{}Matrix Representations of Linear Transformations}

\section*{4.3\quad{}Similarity}

\section*{5.2\quad{}Orthogonal Subspaces}

\end{document}